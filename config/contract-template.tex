%{{=<< >>=}}

\documentclass[]{scrartcl}

% German
\usepackage[ngerman]{babel}

\usepackage{fontspec}
\usepackage{lmodern}

% Euro sign
\usepackage{eurosym}

% Need more space
\usepackage[top=2cm, left=3.5cm, right=3.5cm, bottom=3.6cm]{geometry}

% Header and footer
\usepackage{fancyhdr}
\setlength{\footskip}{35pt}
\fancyhfoffset{0.7cm}
\renewcommand{\headrulewidth}{0.4pt}
\renewcommand{\footrulewidth}{0.4pt}
\pagestyle{fancy} {
  \fancyhead[L]{Kreditvertrag Nr. <<contract.nr>>}
  \fancyhead[R]{Seite \thepage\ von 2}
  \fancyfoot[L]{\footnotesize
    Kto.-Nr.: 1020153845 \\
    BLZ: 120 300 00 \\
    Deutsche Kreditbank AG \\
    IBAN: DE40 1203 0000 1020 1538 45 \\
    BIC: BYLADEM1001
  }
  \fancyfoot[R]{\footnotesize
    Geschäftsführung: \\
    Franziska Heine, Julian Irlenkäuser \\
    Amtsgericht Charlottenburg \\
    HRB 150913 B
  }
  \fancyfoot[C]{\footnotesize
    Hauswärts GmbH \\
    Marchlewskistr. 101 \\
    10243 Berlin
  }
}
\newcommand{\addressStreet}{Burgemeisterstr. 17/18}
\newcommand{\addressWithZIP}{\addressStreet, in 12103 Berlin}

% Contract environment
\usepackage{scrjura}

% Title params
\title{\vspace{-10pt}Kreditvertrag Nr. <<contract.nr>>\vspace{-30pt}}
\date{}

\newcommand*{\yearlyInterestTo}{<<contract.yearlyInterestTo>>}
\newcommand*{\cancelationPeriod}{<<contract.cancelationPeriod>>}
\newcommand*{\interest}{<<contract.interest>>}
\newcommand*{\zeroInterest}{0}

\begin{document}
\maketitle \thispagestyle{fancy}

\noindent Zwischen -Darlehensgeber/in- \\ \\
 \textbf{<<loaner.name>>\\ <<loaner.address>>}  \\ \\
und der <<debtor.name>>, <<debtor.address>> als Darlehensnehmerin wird folgender Vertrag geschlossen:

% Vertragstext beginnt
\begin{contract}

\Paragraph{title={Darlehensbetrag}}

Die <<debtor.name>> erhält ein Darlehen in Höhe von \textbf{\EUR{<<contract.value>>}} (in Worten: <<contract.valueInWords>>). \\
Ändert sich die Darlehenssumme durch weitere Einzahlungen oder Teilrückzahlungen, so behalten die
übrigen Vertragsvereinbarungen ihre Gültigkeit.

\Paragraph{title={Einzahlung}}
Der Darlehensbetrag wird unter Angabe der Vertragsnummer »<<contract.nr>>« überwiesen auf das Konto der <<debtor.name>>.

\Paragraph{title=Verzinsung}
Das Darlehen wird  mit \interest\,\% jährlich verzinst. \\
\ifx\yearlyInterestTo\empty
  \ifx\interest\zeroInterest
  \else
  Die Zinsen werden kumuliert und mit Rückzahlung des Darlehensbetrags ausgezahlt.
  \fi
\else
Die Zinsen werden zum Jahresende dem folgenden Konto gutgeschrieben: \\ \yearlyInterestTo \\
\fi

\Paragraph{title={Kontomitteilung}}
Jeweils nach Ablauf eines Kalenderjahres erhält der/die Darlehensgeber/in eine Mitteilung über den
Kontostand, Ein- und Auszahlungen und über Zinserträge.

\Paragraph{title=Kündigungsfrist}
Das Darlehen wird
\ifx\cancelationPeriod\empty
 befristet gewährt bis zum <<contract.grantedUntil>> (Datum).
\else
 mindestens für <<contract.minimumTerm>> Monate (Mindestlaufzeit) unbefristet gewährt mit einer Kündigungsfrist von \cancelationPeriod\ Monaten.
\fi

\Paragraph{title=Zweck}
Das Darlehen wird verwendet zum Kauf und zur Sanierung des Hauses \addressWithZIP, um dies sozialgebunden zu vermieten und zu verwalten. Des Weiteren dient es zur Finanzierung anderer anfallender Kosten der <<debtor.name>> bezüglich des Hauses und der Verwirklichung des Hausprojektes in der \addressStreet. Die niedrigere Verzinsung des Darlehens als marktüblich ermöglicht tragbare, soziale Mietpreise.

\Paragraph{title=Rangrücktrittsklausel}
»Die Rückzahlung der Darlehen und die Zahlung von Zinsen kann nicht verlangt werden, solange der Darlehensnehmer dieses Kapital zur Erfüllung seiner (nicht nachrangigen) fälligen Verbindlichkeiten benötigt, d.h. es handelt sich um nachrangige Darlehen. Die Darlehensgeber können ihren Anspruch auf Rückzahlung der Darlehen und auf die Auszahlung von Zinsen nicht geltend machen, wenn dies zur Überschuldung oder Zahlungsunfähigkeit des Darlehensnehmers führt. Auch im Insolvenz- oder Liquidationsfall treten die Darlehensgeber mit ihrer Darlehensforderung im Rang hinter die Forderungen aller Gläubiger zurück. Die Rückzahlung des Darlehens kann insofern vom Darlehensnehmer nicht garantiert werden, d.h. es handelt sich nicht um einen unbedingten Rückzahlungsanspruch.« (Diese Klausel ist eine Anforderung der Bundesanstalt für Finanzdienstleistungsaufsicht)

\Paragraph{title=Auflösende Bedingung}
Sollte der Kauf des Hauses \addressWithZIP\ durch die <<debtor.name>> nicht zustande kommen, wird der Kreditvertrag in beiderseitigem Einverständnis aufgelöst und die Darlehenssumme umgehend an den/die Darlehensgeber/in ausgezahlt.

\end{contract}

% Hier kommen die Unterschriten hin
\vspace{1,5 cm}
\begin{tabular}{p{6cm}}
\dotfill \\
Ort, Datum
\end{tabular}%
\hfill
\begin{tabular}{p{6cm}}
\dotfill \\
Ort, Datum
\end{tabular}%

\vspace{1,5 cm}
\begin{tabular}{p{6cm}}
\dotfill \\
Unterschrift Darlehensgeber/in
\end{tabular}%
\hfill
\begin{tabular}{p{6cm}}
\dotfill \\
Unterschrift Darlehensnehmerin
\end{tabular}%

\end{document}
